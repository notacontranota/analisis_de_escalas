\documentclass[a4paper,12pt]{article}
\usepackage{multicol}
\usepackage{amsmath}
\usepackage{musluatex}
\usepackage[hidelinks]{hyperref}

\begin{document}
\title{Análisis de escalas}
\author{Pablo Herrera}
\date{\today}
\maketitle
\begin{multicols}{2}
\begin{abstract}
El presente artículo expone un método de análisis de escalas musicales a partir de los conceptos de \emph{distancia acústica} y \emph{distancia energética} entre los sonidos y la \emph{pertenencia} de los grados respecto a una tónica. Con él, en una escala dada se pueden develar las alianzas y las tensiones internas entre notas que perfilan características de las músicas surgidas de ella.
\end{abstract}
\tableofcontents
\section{Cercanía y lejanía}\label{sec:cercania-lejania}

  Cuando se habla de distancia entre dos sonidos muy habitualmente pensamos que un intervalo compuesto de más \emph{cents} o semitonos tiene más distancia entre sus componentes que otro intervalo con menos \emph{cents} o semitonos. Por ejemplo, decimos que en una quinta justa, de 702 \emph{cents} o 7 semitonos, hay más distancia entre los sonidos que la conforma que en una segunda mayor, de 204 \emph{cents} o 2 semitonos. O dicho de otro modo, mientras menos diferencia de frecuencia exista entre los sonidos del intervalo, más cercanía hay entre ellos. Sin embargo, cambiando hacia una óptica acústica, si observamos la relación entre \emph{do} y \emph{sol} respecto a los armónicos en común entre ambas notas, y hacemos lo propio entre \emph{do} y \emph{re}, notaremos que armónicos más cercanos a las fundamentales (tercer armónico de \emph{do} y segundo de \emph{sol}) son el invisible hilo que une a esas notas, mientras que entre \emph{do} y \emph{re} (noveno armónico de la nota grave y octavo de la nota aguda) ese hilo, construido con armónicos menos audibles y con menor intensidad, hace que la relación de segunda mayor sea \emph{más lejana} que la relación de quinta justa.

  Es más común pensar las distancias entre sonidos del primer modo porque se hace cotidiano, tanto al cantar como al tocar la mayoría de instrumentos musicales, que moverse de \emph{do} a \emph{re} requiere, por lo general, menos energía que la necesaria para moverse de \emph{do} a \emph{sol}. Distancia energética y distancia acústica son dos cosas diferentes: mientras energéticamente \emph{re} está más cerca de \emph{do}, acústicamente \emph{sol} es, de hecho ---luego del unísono y la octava---, la nota más cercana a \emph{do}. \emph{Distancia energética} y \emph{distancia acústica} son los dos conceptos que están en la base del análisis de escalas. Mientras la cercanía energética es la que define uno de los dos arquetipos musicales más importantes, la escala, la cercanía acústica da lugar al otro arquetipo musical sobresaliente, el arpegio. En un arpegio mayor, fundamental, quinta y tercera conforman la tríada de notas más cercanas entre sí\footnote{Aunque entre la tercera y la quinta no existe una armonía donde una sea la fundamental de la otra, sino que ambas tienen en la fundamental de la tríada mayor su posibilidad de estabilidad. Una tercera menor no se estabiliza sino con la presencia de su fundamental.}, y en la escala se manifiesta el movimiento melódico en el que se gradúan los cambios energéticos necesarios para llevar a plano sensible sonidos sucesivos. O dicho de otro modo, la escala representa el movimiento melódico donde se expresa en su orden la mayor cercanía energética entre los sonidos que la componen, y el arpegio es el conjunto de notas acústicamente más cercanas entre sí.

\section{Tipos de movimientos melódicos}\label{sec:movimientos}

  Existen tres formas posibles de relación entre dos notas:
  \begin{enumerate}
    \item Fundamental con un representante de un armónico de dicha fundamental.
    \item Un representante de un armónico con su fundamental.
    \item Dos notas que ninguna es representante de un armónico de la otra.
  \end{enumerate}

  Cuando una fundamental se dirige a una nota representante de un armónico suyo, el efecto sonoro es el contrario al de un movimiento cadencial. Es un movimiento melódico que invita a la continuación, una no-cadencia, una representación de lo abierto.

  Por ejemplo, \\ \musncp{c'1 g' \bar "||" c' e' \bar"||" c'' g' \bar "||" c'' e' \bar "||"}  son movimientos melódicos abiertos, no-cadenciales.

  Cuando una nota representante de un armónico de una fundamental se dirige a ella, se genera la sensación opuesta a la del caso anterior, es decir es un movimiento melódico cadencial, una representación de lo cerrado.

  Por ejemplo, \\ \musncp{g'1 c' \bar "||" e' c' \bar "||" g' c'' \bar "||" \mark "*)"e' c'' \bar "||"} son movimientos melódicos cerrados, cadenciales.\footnote{La sexta menor, marcada con *) en el ejemplo, es acúsiticamente cercana mas no energéticamente, por lo que su poder cadencial queda comprometido.}

  Cuando las notas intervinientes en el movimiento melódico no son una fundamental de la otra, ocurre la mayor de las libertades melódicas, ya que no hay ninguna dependencia entre las notas intervinientes.

  Por ejemplo, \\ \musncp{e'1 g' \bar "||" g' e' \bar "||" e'' g' \bar "||" g' e'' \bar "||" }  son movimientos melódicos libres, semiabiertos o semicerrados, según el contexto. Esto es así porque \emph{mi} no es armónico audible de \emph{sol} y viceversa.

  Una aclaración final antes de pasar al análisis de escalas propiamente dicho: cuando una nota es representante de un armónico de otra con la que está conformando intervalo, decimos que ella tiene \emph{pertenencia} a esa otra nota.

\section{Pertenencia a la tónica de los grados de la escala}\label{sec:pertenencia-grados}
  Una escala con centro tonal es un conjunto de notas comprendidas en una octava que se relacionan con la tónica según su pertenencia o no a la serie de armónicos de dicha tónica. Expresado de otro modo, la tónica es, en cierta forma, la fundamental de la escala.

    \subsection{Una escala sin tensiones internas}\label{subsec:escala-sin-tensiones}
    La escala \hbox{\musncp{\relative{c'2 d4 e g a c2}},} por ejemplo, tiene en \musncp{d'} al armónico 9 de \hbox{\musncp{c'};} en \hbox{\musncp{e'},} al armónico 5 de la tónica; en \hbox{\musncp{g'},} al armónico 3; y en \hbox{\musncp{a'},} al décimo tercer armónico. Esto es importante porque cuando una nota es la representante de un armónico audible de la tónica, tiene, salvo en un caso ---el armónico 11---, la facultad de generar un movimiento cadencial hacia el centro tonal establecido. En este ejemplo, las notas \emph{re, mi, sol, la} son todas capaces de producir un movimiento melódico que produce en la percepción humana la sensación de cierre, de conclusión.
\end{multicols}

\begin{figure}[ht]
\centering
\begin{lilypond}[notime]
\relative {
  \mark \markup{\italic a)}
  d'1 c \mark \markup{\italic b)} \bar "||"
  e c \mark \markup{\italic c)} \bar "||"
  g' c \mark \markup{\italic d)} \bar "||"
  a c \bar "||"
}
\end{lilypond}
\caption{Movimientos melódicos cadenciales de notas con pertenencia a la tónica.}\label{fig:mov-cad}
\end{figure}

\begin{multicols}{2}
  El ejemplo \emph{a)} de la Figura~\ref{fig:mov-cad} muestra un movimiento cadencial muy fuerte, ya que la cercanía energética de \emph{re} a \emph{do} sumado a la relativa cercanía acústica (9:8) confluyen para producir tal fuerza.

  Los ejemplos \emph{b)} y \emph{c)} muestran los movimientos melódicos cadenciales propios de notas representantes de armónicos audibles cercanos dirigiéndose hacia la nota representante de su fundamental.

  Se podrá objetar que en \emph{d)} de la Figura~\ref{fig:mov-cad}, al ser \emph{la} un representante del armónico 13 de la tónica y al ser éste poco audible no hay razón para considerar al movimiento \musncp{a' c''} como cadencial. Sin embargo, tanto la tercera menor ascendente como la descendente son posibles movimientos cadenciales por cercanía energética. Se podría decir que ni \emph{la} es armónico de \emph{do} ni \emph{do} de \emph{la} y por lo tanto es un movimiento melódico libre, y lo es, pero el contexto de haber definido a \emph{do} como tónica convierte a éste en un movimiento cadencial.

    \subsection{Tensión interna en una escala}\label{subsec:tension}

    Tomemos ahora como modelo escalístico a analizar a la escala \hbox{ \musncp{\relative{c'2 d4 e f g a b c2}}.} En esta escala, tras comprobar la pertenencia de los grados a la tónica \emph{do}, notamos que uno de ellos carece tanto de cercanía acústica como de cercanía energética respecto a la tónica: el \grado{4} grado.
\end{multicols}

\begin{figure}[ht]
\centering
\begin{lilypond}[notime]
\score {
  <<
    \new Staff {
      \relative {
        \hide Stem
        c'2 d4 e f g a b c2
      }
    }
    \new Lyrics \lyricmode { "1"2 "9"4 "5" - "3" "13" "15" "2" }
  >>
}
\end{lilypond}
\caption{Pertenencia a la tónica de los grados de la escala de Do mayor. }\label{fig:pertenencia-do}
\end{figure}

\begin{multicols}{2}
    En rigor de verdad, el \grado{4} grado no es lejano acústicamente de la tónica, pero \emph{do} no es la tónica de \emph{fa}, sino al revés: la tónica de la tonalidad es el tercer armónico del \grado{4} grado, algo así como que \emph{fa} es la tónica de \emph{do}. Por esto hay conflicto entre el centro tonal establecido y su \grado{4} grado, porque éste tiende a convertirse en una nueva tónica, compitiendo directamente por el poder dentro de la escala con el \grado{1} grado.

    \subsection{Una escala con más tensiones internas}\label{subsec:esc-mas-tensiones}
    Habiendo visto brevemente las escalas pentatónica y mayor, veamos ahora, también brevemente, al modo menor. La escala \musncp{\relative{\key c \minor c'2 d4 es f g aes bes c2}} con su \grado{7} grado mutable con fines cadenciales y su \grado{6} mutable también, principalmente con fines de conectividad melódica entre el \grado{5} y \grado{7}, posee más que los dos semitonos que tiene la escala mayor, \emph{y donde hay semitonos pueden pasar cosas importantes}. A los semitonos existentes entre \grado{2} y \grado{3} y entre \grado{6} y \grado{5} se suma, vía alteración ascendente, el semitono entre \grado{7} y \grado{8} (o \grado{1}) \hbox{\musncp{\key c \minor b' ( c'' )}.} Al ser el \grado{3} una nota que no representa a un armónico audible de \emph{do}, él no tiene pertenencia a la tónica, no es, por así decirlo, pariente de la tónica; sin embargo el \grado{1} arma alianza con el \grado{3}, junto al \grado{5} ---y con él indirectamente también con el \grado{7} ascendido, aliado natural del \grado{5} por ser representante de su armónico 5--- para conformar el grupo de notas que adhieren al \grado{1} como tónica de la escala. Pero antes de seguir hablando del \grado{3} grado, realicemos el análisis de pernetencia de los grados de la escala menor respecto a su tónica:
\end{multicols}

\begin{figure}[ht]
\centering
\begin{lilypond}[notime]
\score {
  <<
    \new Staff \relative c' {
        \key c \minor
        \hide Stem
        c2 d4 ( es ) f g ( aes ) bes \bar"!" a4 b ( c2 ) \bar "||"
    }
    \new Lyrics \lyricmode {
      "1"2 "9"4 "-" "-" "3" "-" "7" "13" "15" "2"2
    }
  >>
}
\end{lilypond}
\caption{Pertenencia de los grados del modo menor a la tónica.}\label{fig:pertenencia-menor}
\end{figure}

\begin{multicols}{2}
    Como se puede observar en la Figura~\ref{fig:pertenencia-menor}, los grados \grado{3}, \grado{4} y \grado{6} carecen de pertenencia a la tónica, por lo que son potenciales rivales de ella en la lucha por ocupar el centro del sistema. Ahora sí, dicho esto, podemos seguir hablando del \grado{3} grado.

    El semitono \musncp{\key c \minor d' ( es' )} hace del \emph{mi\bemoltxt}, sumado a la no pertenencia de esa nota a la tónica, un potencial centro tonal al cual se llega sin esfuerzo alguno. Todas las condiciones están dadas para que el \grado{3} grado se convierta en el nuevo rey: la tónica no tiene sino inventada por alteración ficta una sensible, una nota energéticamente cercana que se dirija a ella, y el \grado{3} grado sí. Además, el \grado{7} y el \grado{5}, naturales aliados suyos, están también presentes en la escala para conformar el grupo de notas \musncp{es'2 g' bes'} que juntas empoderan a \emph{mi\bemoltxt}.

    Pero el \grado{3} grado no las tiene todas consigo para llegar al poder tonal: el \grado{6} grado, el cual también tiene en sus alrededores un semitono que lo favorece, \hbox{ \musncp{g' ( aes' )},} y para colmo de bienes para él, es el representante de la fundamental de \emph{mi\bemoltxt}. Lo acá descripto no es otra cosa que lo que sucede entre la tónica y el \grado{4} grado en el modo mayor, sólo que en el modo menor, lógicamente por ser tonalidad relativa, ocurre entre \grado{3} y \grado{6}.

    Por último, y no por eso menos significativo en el esquema de poder del modo menor, se encuentra, con todas sus ansias de poder, el \grado{4} grado. Esta vez la anatomía interválica de la escala no lo favorece, como sí sucede en el modo mayor, con un semitono que le conceda mayor peso, pero sigue teniendo, al igual que en el modo mayor, toda la ascendencia sobre el \grado{1} grado.
  \section{Intervalos y afinación}\label{sec:intervalos-afinacion}

  Son intervalos muy diferentes los que se usan estando en una afinación concordante con los armónicos de los sonidos como lo es la \emph{entonación justa} de los que se utilizan estando en una afinación como la del \emph{igual temperamento}. El hecho de llamar \emph{tercera mayor} tanto a un intervalo de proporción $5:4$ como a otro surgido de $2^{\frac{4}{12}}$ es muy significativo y sin dudas la raíz de tratar a esos dos intervalos tan diferentes como si fuesen el mismo intervalo.\footnote{Mientras $5:4=1.25$, $2^{\frac{4}{12}}=1.2599210498948732$, lo que, expresado en \emph{cents}, es $386.3137138648348$ para el intervalo $5:4$ y $400.0$ para el intervalo $2^{\frac{4}{12}}$. La tercera mayor igual temperada es casi 14 \emph{cents} más grande que la tercera mayor de la afinación justa.} En un intervalo $5:4$ hay una dependencia del sonido agudo respecto al grave por ser éste representante de la fundamental de aquél, mientras que en un intervalo $2^{\frac{4}{12}}$ los sonidos que lo conforman carecen de la cercanía acústica que haga de uno de ellos un sonido dependiente del otro; son dos sonidos que se relacionan «en igualdad de condiciones», ninguno de ellos es un representante de un armónico del otro. La libertad de movimiento que tiene todo intervalo dentro del \emph{igual temperamento} es total, ya que no hay movimientos melódicos cadenciales y no-cadenciales ---todos pueden ser cadenciales y todos pueden ser no-cadenciales--- sino solamente movimientos melódicos libres, como los que ocurren entre armónicos de una fundamental ausente. En los \emph{temperamentos irregulares} como los barrocos (\textsc{Vallotti}, \textsc{Wercmeister}, etc.) estamos ante un escenario híbrido: en parte algunos intervalos permanecen puros mientras otros, de misma denominación, son acústicamente lejanos. En \textsc{Kirnberger III} ---uno de los más importantes y usados temperamentos del siglo XVIII--- se dan cuatro tipos de terceras mayores, de los cuales tres de ellas son proporciones que devienen en intervalos conformados por notas independientes, quedando únicamente una de las doce terceras mayores posibles como proporción $5:4$: la tercera mayor de la tríada de Do mayor. Escenario híbrido, pero ya bastante cerca de lo que plantea el \emph{igual temperamento}: la muerte de las relaciones interválicas conformadas por sonidos acústicamente cercanos.

  En el siglo XX se vuelve a poner sobre la mesa el tema del uso de los intervalos compuestos por notas acústicamente cercanas. \textsc{Julián Carrillo} ---compositor romántico y microtonal--- y la \emph{música espectral} dan testimonio de ello.

  En definitiva, las posturas de \textsc{Aristóxeno} ---mentor de la \emph{afinación justa}--- y \textsc{Pitágoras} ---precursor de los temperamentos regulares como el \emph{igual temperamento}--- en relación a la temática de la afinación en música no ha perdido vigencia a través de los tiempos hasta nuestros días, días en los que seguimos debatiéndonos entre las posibilidades combinatorias, las purezas interválicas y la búsqueda de la imposible perfección en un mundo en el que la quietud es utopía.

\section{Estructuras}\label{sec:estructuras}
\end{multicols}

\begin{figure}[ht]
\centering
\begin{lilypond}[notime]
  \layout {
    \context {
      \Voice
      \consists Horizontal_bracket_engraver
    }
  }
  {
    \mark \markup{\italic "a)"}
    \hide Stem
    c'2 e' g' c'' \bar "||"
    \mark \markup{\italic "b)"}
    <<{ \stemDown \hide Stem c''2} \\ {\hide Stem <g'' f'>4}>> \bar "!"
    c''2\startGroup g'4\stopGroup f'\startGroup s c'2\stopGroup \bar "||"
  }
\end{lilypond}
\caption{Dos estructuras basadas en cercanía acústica.}\label{fig:dos-estructuras}
\end{figure}

\begin{multicols}{2}
  \subsection{Arpegio}\label{subsec:arpegio}
  La cercanía acústica de los grados \grado{5} y \grado{3} respecto al \grado{1} propone, en una música que tenga como notas estructurales a estos tres grados, un juego bidireccional de los movimientos melódicos: abiertos cuando la línea melódica se dirige desde \grado{1} hacia \grado{3} o \grado{5}, y cerrados cuando la línea melódica direcciona sus movimientos desde los grados \grado{5} y \grado{3} hacia el \grado{1} grado. Mientras los movimientos melódicos se produzcan entre \grado{3} y \grado{5} el transcurrir melódico nos deposita en una zona de no-definición, de prolongación de lo abierto e irresuelto. Una baguala\footnote{Una baguala es una pieza musical característica de la zona de montaña del norte argentino (provincias de Salta y Jujuy) de carácter improvisatorio basada en una escala tritónica donde fundamental, quinta y tercera juegan acompañadas de una caja, instrumento de percusión también típico de esta región.}, que además de usar las notas de la estructura arpegio excluye a otras notas, desnuda y revela con más facilidad lo anteriormente dicho: \mus{\time 3/4 \relative c' {c4. 8 e4 g4. e8 g4 g4 e2}} es una frase que, estructuralmente se hace \musncp{\hide Stem c'2 e' g' e'} es decir un movimiento no-cadencial \musncp{c'2 e'} seguido de una prolongación de lo abierto en \musncp{e'2 g' e'} haciendo de toda la frase musical un fragmento de carácter abierto, irresuelto; el consecuente habitual de esta frase suele ser, típicamente, \mus{\time 3/4 \relative g'{g4. e8 g4 4. e8 c4 4 2 \bar "|."}} que cierra con el movimiento melódico cadencial \musncp{e'2 c'} el cual es antecedico por la prolongación de lo abierto con las notas\hbox{\musncp{g'2 e' g' e'}.}

  El fragmento \musncp{c'2 f'4 d' e'2 a'4 g'2}, desarrollo melódico ya dentro de un ámbito heptatónico, no es disímil en términos estructurales al antecedente del ejemplo de baguala precedente: \musncp{c'2 e' g'} siguen siendo las notas que sostienen ese tejido melódico y, por ser movimientos melódicos no-cadenciales, este fragmento también posee carácter abierto, no resolutivo. Si completamos la frase anterior con un posible fragmento como \musncp{g'2 f'4 e'2 d'4 b c'2} estamos, una vez más, ante una situación análoga al consecuente de la baguala: una estructura melódica \musncp{g'2 e' c'} que completa y cierra, por su carácter cadencial, la frase musical.

  Lo tritónico subyace en lo pentatónico, y lo pentatónico subyace en lo heptatónico. Las escalas analizadas brevemente en la sección~\ref{sec:pertenencia-grados} son las que son justamente por la importancia estructural e histórica que ellas tienen en la música.

  \subsection{Tetracordios}\label{subsec:tetracordios}

  La cercanía acústica de los grados \grado{4} y \grado{5} respecto a la tónica \grado{1} es algo que se ve reflejado en innumerables manifestaciones culturales distantes entre sí tanto en tiempo como en espacio. Una de esas manifestaciones, repetidas en múltiples culturas, consiste en las maneras de cantos \emph{antifonales} y \emph{responsoriales} encontrables en trabajos comunales, organizaciones militares o en ritos religiosos. El ejemplo \emph{b)} de la Figura~\ref{fig:dos-estructuras} representa, enmarcada en el \emph{diapasón}, la estructura basamentada tanto en el pensar a toda escala como un descenso hacia la tónica como en encontrar en sus dos notas más cercana a la tónica los puntos de apoyo para el desarrollo en el tiempo de las fuerzas internas de una escala que contenga a esas dos notas.

\section{Simetrías}\label{sec:simetrias}
Los espejos han fascinado y fascinan a los seres humanos desde tiempos inmemoriales. La presencia de ellos en la música es más que notoria y constituye causal de estructuras fundamentales y generadora de nuevas sonoridades permanentemente.  Los tipos de espejos más comunes son el vertical y el horizontal: mientras el espejo vertical refleja horizontalmente, el espejo horizontal hace lo propio verticalmente. No importa qué refleje, al hacerlo se produce, en la unión del original y su reflejo, un objeto simétrico. Este objeto es de carácter melódico cuando él es desplegado en el tiempo y de carácter armónico cuando se concentra en un momento. O dicho de otro modo: tanto a objetos melódicos como a objetos armónicos, aunque ellos sean asimétricos, al aplicarle alguno de estos espejos (vertical, horizontal o la combinación de ambos) obtenemos, en la combinación de original e imagen espejada, un objeto más complejo que es siempre simétrico. El arpegio mayor hbox{\musncp{c''2 e''},} por tomar un ejemplo melódico, al aplicarle un espejo vertical obtenemos el objeto melódico más complejo \hbox{\musncp{c''2 e'' \bar "!" e'' c''};} al aplicarle un espejo horizontal al mismo objeto melódico, obtenemos \hbox{\musncp{c''2 e'' \bar "!" c'' aes'}.} Las notas acá participantes de los objetos, al ser utilizadas en la simultaneidad, conforman objetos que en su significado musical más profundo no son muy diferentes, es decir \musncp{c''2 e''} no es muy diferente a \hbox{\musncp{<e' c''>2},} y \musncp{e''2 c''} no difiere en demasía en el significado respecto a \hbox{\musncp{<c'' e''>2},} ya que \nota{do}{5}-\nota{mi}{5} es un movimiento melódico abierto y el acorde \nota{mi}{4}-\nota{do}{5} es un intervalo que no reposa y por lo tanto también de carácter abierto. En cambio, \musncp{e''2 c''} es un movimiento melódico cerrado, cadencial, y \musncp{<c'' e''>2} es un acorde también cerrado, con la fundamental en el bajo y un representante de un armónico propio en el agudo. Este es el motivo por el cual consideramos que la versión espejada de \musncp{<c'' e''>2} puede ser \musncp{<e' c''>2}, por ser uno el reverso semántico del otro, análogo a \musncp{c''2 e''} y \hbox{\musncp{e''2 c''}.} También es cierto que el reflejo de \musncp{<c'' e''>2} puede ser \hbox{\musncp{<aes' c''>2}.} ¿Cuál, entonces, refleja a \hbox{\musncp{<c'' e''>2}?} ¿\hbox{\musncp{<e' c''>2},} o \hbox{\musncp{<aes' c''>2}?} En un contexto de afinación natural, \hbox{\musncp{<e' c''>2},} y en un contexto de afinación igual temperada, \hbox{\musncp{<aes' c''>2}.} Los espejos de sujeto y respuesta «tonal» en las fugas del siglo XVIII ---y no únicamente en ellas---, que suponen \musncp{c'2 g' \bar "!" g' c''} confirman la necesidad de incluir en las imitaciones el factor semántico de los intervalos, siempre que se esté en un contexto de afinación natural. El siglo XVIII europeo es, en este sentido, un tiempo de experimentación y de transición desde la afinación natural hacia la afinación temperada, donde la práctica compositiva estaba más cerca del pensamiento musical surgido desde la relación de los sonidos con sus armónicos más cercanos que de la especulación sonora que se practica con los dos pies puestos sobre un temperamento igual.

\section{Movimientos}\label{sec:movimientos}
Quietud, movimiento, intervalos melódicos y armónicos son lo mismo. Paralelismos entre voces como quietud; paralelismos como engrosamiento de una voz.


\end{multicols}


\begin{multicols}{2}
\listoffigures
\end{multicols}
\end{document}
